%\documentclass{report} % o article, book, memoir, etc.
%\usepackage[a4paper, left=3cm, right=3cm, top=2.5cm, bottom=2.5cm]{geometry}
\documentclass[12pt]{report}
\usepackage[a4paper, left=2.5cm, right=2.5cm, top=2.5cm, bottom=2.5cm]{geometry}
\usepackage{lmodern} % Font simile a quello della classe book
\usepackage{setspace} % Per impostare l'interlinea
%\onehalfspacing 
%\documentclass[12pt,a4paper,twoside,titlepage]{book}

%\usepackage[a4paper,width=130mm,top=30mm,bottom=30mm]{geometry} 

%%%%%%%%%%%%\usepackage{fancyhdr}

% Imposta un'altezza della testa adeguata
%%%%%%%%%%%%\setlength{\headheight}{14.5pt} % Puoi regolare il valore di 14.5pt secondo il suggerimento

% Eventualmente, puoi regolare il margine superiore per compensare
% \addtolength{\topmargin}{-2.5pt} % Puoi regolare il valore di -2.5pt secondo il suggerimento

% Pulisci tutti i campi di testa e di piede
%%%%%%%%\fancyhf{}

% Definisci un nuovo stile di intestazione
%%%%%%%%\fancypagestyle{mystyle}{%
%%%%%%%%%%%    \fancyhead[LE,RO]{\thepage} % Numero di pagina in alto a destra nelle pagine dispari e a sinistra nelle pagine pari
 %%%%%%%%%%%   \fancyhead[LO,RE]{\leftmark} % Titolo del capitolo a sinistra nelle pagine dispari e a destra nelle pagine pari
%%%%%%%%%%%5}

% Rimuovi il "Chapter" dal titolo del capitolo nell'intestazione
\renewcommand{\chaptermark}[1]{\markboth{#1}{}}

% Applica lo stile personalizzato alle pagine desiderate
%%%%%%%%%%\pagestyle{mystyle}
% Definisci un nuovo tipo di colonna con larghezza fissa e contenuto centrato


\usepackage[utf8]{inputenc}
\usepackage[T1]{fontenc}
\usepackage{lmodern}          % Utilizza font scalabili Latin Modern
\usepackage{microtype}
 % Per migliorare la qualità della tipografia
\usepackage{comment}
\usepackage{graphicx}
\usepackage{multirow}
\usepackage{array} % Pacchetto per personalizzare le colonne nelle tabelle

\usepackage[rgb,dvipsnames]{xcolor}
\usepackage[small]{caption}
\usepackage{listings}

\definecolor{codegreen}{rgb}{0,0.6,0}
\definecolor{codegray}{rgb}{0.5,0.5,0.5}
\definecolor{codepurple}{rgb}{0.58,0,0.82}
\definecolor{backcolour}{rgb}{0.95,0.95,0.92}
\lstdefinestyle{mystyle}{
    backgroundcolor=\color{backcolour},   
    commentstyle=\color{codegreen},
    keywordstyle=\color{magenta},
    numberstyle=\tiny\color{codegray},
    stringstyle=\color{codepurple},
    basicstyle=\ttfamily\footnotesize,
    breakatwhitespace=false,         
    breaklines=true,                 
    captionpos=b,                    
    keepspaces=true,                 
    numbers=left,                    
    numbersep=5pt,                  
    showspaces=false,                
    showstringspaces=false,
    showtabs=false,                  
    tabsize=2
}
\lstset{style=mystyle}

\usepackage{float}
\usepackage{booktabs}
\usepackage{parskip}
\usepackage{hologo}
\usepackage{shellesc}
\usepackage{epigraph}
\usepackage{etoolbox}
\usepackage{wrapfig}

\setlength\epigraphrule{0pt}
\setlength\parindent{0pt}
\setlength{\epigraphwidth}{0.5\textwidth}

\usepackage{colortbl}

%\usepackage[toc,page]{appendix}
\usepackage{enumitem}

\usepackage{subcaption}

% Math and physic packages
\usepackage{siunitx}
\usepackage{physics}
% Sovrascrivi la definizione di \qty per utilizzare \SI di siunitx
%\AtBeginDocument{\RenewCommandCopy\qty\SI}
\usepackage{dsfont}
\usepackage{amsthm,amsmath,amssymb,amsfonts,bbm}
\usepackage{mathrsfs}
\usepackage{tensor}
\usepackage{mathtools}
\usepackage{url}

\usepackage{hyperref}
\hypersetup{
    colorlinks=true,
    linkcolor=true, % colore dei link interni
    urlcolor=true   % colore degli URL
}

%\usepackage[a-1b]{pdfx}   % for PDF/A-1b


\definecolor{true}{RGB}{0,0,0}

\newcommand{\epsv}{\varepsilon_{0}}
\newcommand{\todo}[1]{\textcolor{blue}{\textbf{TODO:} #1}}
\newcommand{\gp}[1]{\textcolor{cyan}{\textbf{GP:} #1}}
\newcommand{\mgug}[1]{\textcolor{red}{\textbf{MGUG:} #1}}%
\newcommand{\cf}[1]{\textcolor{brown}{\textbf{CF:} #1}}%
\renewcommand\labelitemi{\tiny$\bullet$}
%%%%%%%%%%%%TITLE
\usepackage[calcwidth,pagestyles]{titlesec}% loads titleps
\usepackage{adjustbox,xcolor}
% chapter head style via titlesec
\titleformat{\chapter}[display]
{\bfseries\Large}
{\color{black!65!black}\filleft%
\minsizebox{!}{24pt}{\chaptertitlename}% needs package adjustbox
\lapbox[2em]{1.5em}{% Modifica la spaziatura a sinistra del "Chapter"
%\lapbox[0pt]{0.8em}{% Modifica la spaziatura a sinistra del "Chapter"
%\lapbox[0pt]{\width}{%
\minsizebox{!}{40pt}{%
\ %%%quadrante
\colorbox{white!50!white}{\color{black}\thechapter}% needs xcolor
}%
}% needs package adjustbox
}
{4ex}
{{\color{black!65!black}\titlerule}
\huge\bfseries\scshape
\vspace{2ex}%
\filright}
[\vspace{2ex}%
{\color{black!65!black}\titlerule}]



% Definisci uno stile per l'intestazione personalizzato
\newpagestyle{mystyle}{
    % Pulisci tutti i campi di testa e di piede
    \sethead{}{}{}
    
    % Linea sopra il numero di pagina e il nome del capitolo
    \setheadrule{0.4pt}
    
    % Numero di pagina in alto a destra nelle pagine dispari e a sinistra nelle pagine pari
    \sethead[\thepage][][\chaptertitle]{\chaptertitle}{}{\thepage}
}

% Applica lo stile personalizzato alle pagine desiderate
\pagestyle{mystyle}
\usepackage[backend=biber,style=numeric,citestyle=numeric]{biblatex}

%\usepackage[style=numeric]{biblatex} % Specifica lo stile bibliografico
\addbibresource{sample.bib} 

%Imports bibliography file
% Documents
\usepackage[swapnames]{frontespizio}


\begin{document}
\begin{titlepage}

    \begin{center}
    {\LARGE Lecture Notes
    }
    \end{center}
    
    \begin{center}
    \vspace{1cm}
    {\huge\textbf{Title to decide. Technical notes for Physics} \par}
    \end{center}
    \par
      \vspace{2 cm}
    
            

\end{titlepage}
\null\thispagestyle{empty}\newpage % Pagina bianca dopo il frontespizio

\thispagestyle{empty}
\null\vspace{\stretch{1}}
\begin{flushright}
\textit{Dedica}
\end{flushright}
\vspace{\stretch{2}}\null
\null\thispagestyle{empty}\newpage % Pagina bianca dopo il frontespizio

\setcounter{tocdepth}{4}
\shipout\hbox{} %%%%%%new command

\tableofcontents


\clearpage 



\chapter{Introduction}

In this work Niccolò and Cecilia want to create lecture notes
 on the language program Pyhton \footnote{\href{https://www.python.org/}{Python}}
with a deep focus on Pytorch\footnote{\href{https://pytorch.org/docs/stable/index.html}{Pythorch documentation}}
  and TensorFlow\footnote{\href{https://www.tensorflow.org/api_docs}{TensorFlow documentation}}.
 Sections designated to plots and basics statistical methods will also be provided...


 \section{How to write code text in LateX}

 If you want to write codes from a .py in a specific box, you need to  use the following command:
 lstinputlisting[language=Octave]{$example_code.py$}
 \lstinputlisting[language=Octave]{example_code.py}






\chapter{Plotly}
\label{sec: plotly}

\section{Boxplots}
Boxplots


\begin{lstlisting}[language=Python, caption=Python example]
import plotly.express as px
import plotly.graph_objects as go

df = pd.read_excel(input)
fig_asd = px.box(df, y=['ASD [mm]'], points="all")
fig.add_trace(go.Box(fig_asd['data'][0], marker_color='darkblue', boxmean=True))
fig.update_layout(title='',showlegend=False, boxgap=0.1, boxgroupgap=0.5, font=dict(size=25))
fig.write_html(os.path.join(output_path_folder, 'bb_metric.html'))
# to run this Kaleido need to be installed in a environment path without any accent
fig.write_image(os.path.join(output_path_folder, 'bb_metric.png')) 

fig.show()

\end{lstlisting}

\chapter{Statistical methods}
\label{sec: statistical }
\section{Linear regression}
\section{Logistic Regression}
\section{Classification}

\section{Principal Component analysis}

\section{Statistical test}

\subsection{ANOVA}

\subsection{ chi double}




\printbibliography[
heading=bibintoc,
title={References}
]

%Filters bibliography
%\printbibliography[type=article,title={Articles}]
%\printbibliography[type=book,title={Books}]
%\printbibliography[keyword={latex},title={Thesis}]

% Pagina bianca dopo il capitolo 2
\clearpage % Nuova pagina

\end{document}
